\documentclass[a4paper,twoside]{report}

\usepackage{graphicx}
\usepackage{dsfont}

\usepackage{epsf,amsthm,amsmath}
\usepackage[ngerman]{babel}
\usepackage[latin1]{inputenc}
\usepackage{graphicx}
\setlength{\parskip}{5pt plus 8pt minus 2pt}
\pagestyle{headings}

%----------------------------------------------------------------------
%  Makros
%----------------------------------------------------------------------
%\usepackage{dsfont}
%\def\C{\mathds{C}}
%\def\F{\mathds{F}}
%\def\R{\mathds{R}}
%\def\N{\mathds{N}}
%\def\Z{\mathds{Z}}
%
%\newcommand{\bra}[1]{\langle#1|}
%\newcommand{\ket}[1]{|#1\rangle}
%\newcommand{\braket}[2]{\langle#1|#2\rangle}
%\newcommand{\ketbra}[2]{|#1\rangle\langle#2|}
%\newcommand{\projektor}[1]{|#1\rangle\langle#1|}
%\newcommand{\schnitt}[2]{
%   \raise3pt\vbox{\moveright6.5pt\hbox{$#1$}}\hspace*{-4pt}\Bigm/
%   \lower3pt\vbox{\moveleft6.5pt\hbox{$#2$}}}
%
%\newcommand{\includeeps}[2]{
%   \def\epsfsize##1##2{#2##1}
%   \centerline{\epsffile{#1}}}
%
%\newtheorem{satz}{Satz}[chapter]
%\newtheorem{bem}[satz]{Bemerkung}
%\newtheorem{lem}[satz]{Lemma}
%\newtheorem{defi}[satz]{Definition}
%\newtheorem{beispiel}[satz]{Beispiel}
%----------------------------------------------------------------------

\begin{document}
\begin{titlepage}
\ \vfill
\Large
\begin{center}
{\LARGE\bf Seminar} \\[1cm]
{\huge\bf Biologische Prinzipien in der Informatik\par}
\vspace*{1cm}
\input unilogo
\unilogo{30}\\[1cm]
{\bf Universit"at Karlsruhe (TH)}\\
{Fakult"at f"ur Informatik}\\
{\em Institut f"ur Prozessrechentechnik, Automation und Robotik}
\vfill
Prof. Dr. rer. nat. Brinkschulte\\
Manuel Nickschas\\
Florentin Picioroaga\\
Mathias Pacher\\
Sebastian Schuster\\
Alexander von Renteln	  
\vfill\vfill 
Sommersemester 2006
\vfill
\vfill
\end{center}
\end{titlepage} 
% 
%\newpage
%\thispagestyle{empty}
%\ 
%\newpage 
\thispagestyle{empty}
\ 
\vfill
\noindent
Copyright $\copyright$ 2006\\
Institut f"ur Prozessrechentechnik, Automation und Robotik (IPR)\\
Abteilung Prof. Dr. rer. nat. Brinkschulte (Mikrorechnertechnologien f"ur Automatisierung)\\
Geb"aude 40.28 - Engler-Bunte-Ring 8\\
76\,131 Karlsruhe
 
%\newpage 
%\thispagestyle{empty}
%\ 
%\newpage 



\title{Humorale Immunantwort}
\author{Clemens Lode}

\date{}
\maketitle

%\thispagestyle{empty}
%\newpage 

\pagenumbering{roman}
\tableofcontents

%\newpage 
\pagenumbering{arabic}


\setcounter{chapter}{0}
\chapter{Humorale Immunantwort}
\section{Einleitung}

\newpage
\section{Grundeigenschaften des Immunsystems}
\subsection{Grundmerkmale}

Antigenpräsentierende Zellen (APC, z.B. Makrophagen) durchqueren den Körper,

verspeisen Antigene die sie finden und fragmentieren sie. Aus den Fragmenten
entstehen sog. MHC- Moleküle (major histocompatibility complex), die an der
Oberfläche der APC präsentiert werden.


- Eine T-Helfer-Zelle mit passendem Rezeptor f"ur das pr"asentierte MHC-Molek"ul kann nun an der Makrophage andocken und wird dadurch aktiviert.
- Die aktivierte Zelle teilt sich und sch"utet Chemikalien (Lymphokine) aus, die wiederum B-Zellen anlocken.
- Eine B-Zelle mit passendem Rezeptor reagiert auf die ausgesch"uteten Chemikalien und kann direkt an das Antigen andocken (in diesem Fall ohne MHC-Molek"ul), fragmentiert es und pr"asentiert es ebenfalls an der Oberfl"ache. Die aktivierte T-Helferzelle dockt an das pr"aentierte Antigen der B-Zelle an und aktiviert diese.
Die aktivierte B-Zelle teilt sich in Plasmazellen, die die Antikörper (Kopien des
   Rezeptors) ausschütten.
Die Antikörper können nun die Antigene neutralisieren und für die Phagozyten
   markieren.

\subsection{Impfung}
Neben dem nat"urlichen Weg einer Infektion (z.B. Tr"opfcheninfektion) besteht auch die M"oglichkeit das eigene Immunsystem mittels einer Impfung mit der Krankheit zu konfrontieren. Dabei werden abgeschw"achte Erreger (inaktivierte Bakterientoxine oder Viren, abget"otete oder geschw"achte Bakterien oder auch isolierte Makromolek"ule bzw. deren Fragmente) gespritzt oder eingenommen.

~Konservierungsstoffe und koerperfremde Eiweisse.
~Impfschaden

\subsection{Aktive und passive Immunit"at}

Aktive und passive Immunit"at betrifft die Herkunft der Antik"orper~~. Aktive Immunit"at hat die Person, deren Immunsystem die Antik"orper selbst, als Antwort auf die Pr"asenz von Pathogenen~~, produziert hat. Dabei spielt es keine Rolle ob diese Pr"asenz nat"urliche Gr"unde hat oder k"unstlich mit z.B. einer Impfung herbeigef"uhrt wurde.
Passive Immunit"at bezeichnet den direkten "Ubergang von Antik"orpern entweder von einem Menschen auf den anderen (z.B. bei der Schwangerschaft, von der Mutter zum Kind) oder mithilfe einer Passivimpfung, bei der nur die Antik"orper und nicht die Antigene gespritzt / eingenommen werden. Passive Immunit"at h"alt sich jedoch lediglich f"ur einige Wochen oder Monate, das k"orpereigene Immunsystem kennt die Antigene nicht und kann selbst auch keine Antik"orper dagegen produzieren, es kann nur von den Antigenen direkt lernen.

Der Sinn einer aktiven Immunisierung liegt prim"ar darin, dass bei einer zuk"unftigen Infektion mit dem Antigen das Immunsystem sofort reagieren kann. Der Sinn einer aktiven Immunisierung ist der sofortige Schutz vor einem Antigen bis das k"orpereigene Immunsystem das neue Antigen gelernt hat und eigene Antik"orper produzieren kann.

\subsection{Die sekund"are Immunantwort}
Wird das Immunsystem einmal mit unbekannten Antigenen konfrontiert, dann werden langlebige Ged"achtniszellen angelegt, die sich rasch vermehren, wenn sie erneut mit dem Antigen konfrontiert werden. Diese sekund"are Immunantwort verl"auft wesentlich schneller und die diesmal produzierten Antik"orper binden das Antigen mit h"oherer Affinit"at als jene, die im Verlauf der prim"aren Immunantwort gebildet wurden. 

\subsection{Entwicklung der Lymphocyten}


\subsection{Klonale Selektion}

\newpage
\section{Humorale Immunantwort}

\subsection{B-Zellen}
B-Lymphozyten reifen im Knochenmark 
Bei Kontakt mit einem Fremdk"orper entwickelt sich ein Teil der B-Lymphozyten zu so genannten Plasmazellen, die Antik"orper (=Immunglobuline, Ig) gegen diesen Fremdk"orper bilden. Plasmazellen leben etwa 2-3 Tage. 
Aus dem anderen Teil der B-Lymphozyten werden nach Kontakt mit einem Fremdk"orper langlebige B-Ged"achtniszellen, die noch Jahre sp"ater, auch wenn der K"orper nicht mehr diesem Fremdk"orper ausgesetzt ist, die gleichen Antik"orper bilden k"onnen, was eine schnellere Reaktivierung der adaptiven Immunabwehr induziert.

Jede reife B-Zellen besitzt auf ihrer Oberfl"ache einen f"ur diese Zelle spezifischen Antik"orper, der als Antigenrezeptor fungiert und f"ur den diese Zelle f"ur den Rest ihres Lebens zust"andig ist. 

TODO IG-Typen?
Affinitaetsreifung IgM -> IgG

\subsection{Aktivierung der T- und B-Zellen}
Der erste Schritt bei der Aktivierung der B-Zellen ist die Bindung eines Antigens an einen spezifischen Rezeptor auf der Oberfl"ache der B-Zelle. Um tats"achlich aktiv zu werden, gibt es im menschlichen Immunsystem aber noch einen Zwischenschritt um die Wahrscheinlichkeit fuer Autoimmunkrankheiten, d.h. die Erkennung eigener Zellen als Fremdk"orper, zu minimieren.
Makrophagen, die Fremdk"orper aufgenommen und verdaut haben, pr"asentieren dessen Antigene auf ihrer Oberfl"ache. T-Zellen erkennen diese Antigene durch Andocken an die Makrophage und klont sich selbst zu einer spezialisierten T-Helferzelle die "uber Cytokine selektiv solche B-Zellen stimuliert, die mit einem solchen Antigen bereits zu tun gehabt haben.
B-Zellen pr"asentieren ebenfalls die Antigene auf ihrer Oberfl"ache und T-Zellen erkennen diese ebenfalls auf die selbe Weise. Der einzige Unterschied von Makrophagen und B-Zellen in dem Zusammenhang ist, dass Makrophagen mehrere verschiedene Antigene pr"asentieren k"onnen w"ahrend B-Zellen, wie auch T-Zellen, antigenspezifisch sind.
Sobald ein durch den Kontakt mit einer Makrophage und einer T-Zelle entstandene spezialisierter T-Helferzellenklon an das von der B-Zelle pr"asentierte Antigen angedockt hat, wird die B-Zelle aktiviert. Aktivierte B-Zellen differenzieren dann zu Plasmazellen und langlebigen Ged"achtniszellen.
Plasmazellen produzieren dann w"ahrend ihrer Lebenszeit Antik"orper bis die Infektion vor"uber ist und einige Zeit dar"uber hinaus um eine Zweitinfektion rasch bek"ampfen zu k"onnen.


\subsection{Rolle der Antigene}

QUELLE
Als Antigen wird alles bezeichnet, was eine adaptive Immunreaktion ausl"osen kann. Die chemische Zusammensetzung ist dabei von geringer Bedeutung. Antik"orper erkennen gr"ossere, dreidimensionale Oberfl"achenstrukturen der Antigene. Oft hat ein Makromolek"ul mehrere solcher Strukturen, so genannte antigene Determinanten oder Epitope, die unabh"agig voneinander die Bildung von Antik"orpern ausl"osen. Umgekehrt k"onnen zwei in der Prim"arstruktur ganz unterschiedliche Molek"ule trotzdem vom selben Antik"orper erkannt werden, wenn ihre Oberfl"achenstruktur zuf"allig sehr "ahnlich ist. Man spricht dann von einer Kreuzreaktion. T-Lymphozyten sind nicht auf dreidimensionale Epitope, sondern auf lineare
Peptide von 8 bis 20 Aminos"auren L"ange spezialisiert.

\subsection{Rolle der Antik"orper}
Antik"orper verankern sich direkt mit den Antigenen, die sich auf der Oberflaeche des Fremdk"orpers befinden. Die so markierten Viren, auch "Antigen-Antikoerper-Komplex" genannt, werden dann durch phagocytotische Zellen beseitigt.
Zus"atzlich k"onnen sich die Antik"orper zu Agglutinationen (Verklumpungen) zusammenschliessen, da jedes Antik"orpermolek"ul mindestens zwei Antigenbindungsstellen besitzt und somit benachbarte Antigenmolek"ule vernetzen kann.
So markierte Baktieren k"onnen leichter durch die schon erw"ahnten phagocytotischen Zellen aufgefunden werden.
TODO loesliche Antigenmolekuele?

\newpage

\section{Selbst-Fremd-Erkennung}

Um Fremdk"orper zu erkennen muss nicht nur der K"orper an sich erkannt werden, es muss auch festgestellt werden, ob es sich dabei um eine eigene Zelle oder eine Fremdzelle (oder Fremdk"orper) handelt. Hierzu muss dem Immunsystem zum einen ein Merkmal vorliegen, das die eigenen Zellen eindeutig identifiziert und zum anderen muss dieses Merkmal gelernt werden.
Hierzu muss in kontrollierter Umgebung jede neuentstandene T- und B-Zelle getestet werden, ob sie k"orpereigene Molek"ule erkennt und je nach dem zerst"ort oder funktionsunf"ahig gemacht werden. Dieser Vorgang nennt sich "selektive Selektion" und im K"orper passiert er im Thymus und im Knochenmark.
Das eindeutige Merkmal sind eine Gruppe von Proteinmolek"ule die durch eine Anzahl von Genen codiert werden und "Haupthistokompatibilitaetskomplex" (MHC-major histocompatibility complex) genannt werden. Aufgrund der grossen Anzahl von M"oglichkeiten (20 MHC-Gene und ~100 Allele von jedem Gen) hat jeder Mensch einen unterschiedlichen MHC Marker.

In der Anwendung in der Informatik hiesse das, dass dem kuenstlichen Immunsystem ein 'gesunder' Ablauf in einer Simulation vorgespielt wird und alle Antik"orper die auf diesen Ablauf ansprechen gel"oscht werden. Diese Bedingung muss erfuellt sein, auch das menschliche Immunsystem wuerde nicht funktionieren, wenn die Antik"orper nicht aufgel"ost w"urden, die auf eigene Antigene ansprechen bzw. wenn im Knochenmark selbst Antigene von k"orperfremden und gef"ahrlichen Erregern liegen w"urden.
	
\section{Selbst-Fremd-Erkennung am Beispiel Bluttransfusion}
Blutzellen, wie jede K"orperzelle, haben spezifische Antigene auf ihrer Oberfl"ache. Im Menschen gibt es zwei verschiedene Antigene, A und B. Ein Mensch mit Blutgruppe A hat Blutzellen mit Antigen A und er besitzt Antik"orper zu Antigen B, ein Mensch mit Blutgruppe B hat Blutzellen mit Antigen B und besitzt Antik"orper zu Antigen A. Empf"angt ein Mensch mit Blutgruppe A Blut von einem Menschen mit Blutgruppe B werden diese Blutzellen als Fremdk"orper angesehen und vom Immunsystem angegriffen.
Sonderstellungen haben Menschen mit Blutgruppe 0, deren Blutzellen haben weder Antigene A oder B, sie k"onnen also jedem Menschen spenden. Umgekehrt haben sie aber Antik"orper gegen A und gegen B, k"onnen selbst also nur Blut von Menschen mit Blutgruppe 0 empfangen. Menschen mit Blutgruppe AB k"onnen selbst nur Menschen mit Blutgruppe AB spenden, da sie aber keine Antik"orper gegen A oder B haben, k"onnen sie Blut jeglichen Typs empfangen.

Dass "uberhaupt Antik"orper gegen andere Blut-Antigene existieren, liegt daran, dass die Antigene recht einfach aufgebaut sind und auch bei einigen, ungef"ahrlichen Bakterien auftauchen. Beispielsweise in Menschen mit Blutgruppe A existieren keine Bakterien mit Antigen B, da sie vom Immunsystem unsch"adlich gemacht werden. Gleichzeitig erwirbt das Immunsystem dadurch auch eine Immunit"at gegen Blut mit Antigen B. Bakterien die Antigen A besitzen werden dagegen nicht angegriffen.

Ein Sonderfall ist der sog. Rhesus-Faktor. Ein Embryo hat nicht die selben DNS wie die Mutter, es m"ussen also spezielle Vorkehrungen existieren, damit der Embryo nicht als Fremdk"orper abgestossen wird. Dies geschieht durch die Placenta-Barriere, die auch zur Versorgung des Kindes dient. Durch diese Barriere k"onnen Antik"orper der Klasse IgM nicht hindurch. Andere Antik"orper, gegen Blutzellen mit Antigenen gegen den Rhesus-Faktor der Blutzellen.

Ist die Mutter selbst Rh-positiv, dann besitzt sie keine Antik"orper gegen den Rhesus-Faktor, kann auch keine bilden und es besteht kein Problem. Ist die Mutter selbst Rh-negativ kann es passieren, dass, wenn z.B. bei der Geburt kleine Blutmengen in den Kreislauf der Mutter gelangen, dort Antik"orper gegen den Rhesus-Faktor gebildet werden. Bei einer zweiten Schwangerschaft k"onnen diese gebildeten Antik"orper in den Organismus des Embryos gelangen und dort die Rh-positiven Blutzellen angreifen.

Als Gegenmassnahme kann man der Mutter nach der Geburt durch eine Passivimpfung mit Anti-Rh-Antik"orpern injizieren. Dadurch bildet das m"utterliche Immunsystem keine eigenen Antik"orper und weitere Schwangerschaften sind m"oglich.

\newpage
\section{Umsetzung in der Informatik}

Im folgenden soll eine vereinfachte, auf die Grundprinzipien reduzierte Umsetzung des Immunalgorithmus in der Informatik dargestellt werden. Bei der Umsetzung in der Informatik sind einige die sich im menschlichen K"orper stellenden Probleme irrelevant, beispielsweise m"ussen nicht alle Funktionsbestandteile dezentralisiert sein wie im Immunsystem, der Algorithmus selbst muss nicht parallelisiert werden sondern kann zentral auf einer CPU ablaufen.

\subsection{Interne Darstellung}

Antik"orper und Antigene werden durch Strings dargestellt, die aus drei Symbolen bestehen: '1', '0' und ein 'dontcare'. Trifft ein Antik"orper auf ein Antigen mit invertiertem Bitstring, dann wird das Antigen erkannt. Ein 'dontcare' Symbol passt immer, d.h. ein Antik"orper mit Bitstring ****** w"urde jedes Antigen erkennen.

TODO
selektive Deletion:

In der Anwendung in der Informatik hiesse das, dass dem kuenstlichen Immunsystem ein 'gesunder' Ablauf in einer Simulation vorgespielt wird und alle Antik"orper die auf diesen Ablauf ansprechen gel"oscht werden. Diese Bedingung muss erfuellt sein, auch das menschliche Immunsystem wuerde nicht funktionieren, wenn die Antik"orper nicht aufgel"ost w"urden, die auf eigene Antigene ansprechen bzw. wenn im Thymus bzw. Knochenmark selbst Antigene von k"orperfremden und gef"ahrlichen Erregern liegen w"urden.


W"urde ein Antik"orper alle Antigene erkennen, dann w"are das System zwar 'immun' gegen alle Fremdk"orper, erkennt sich aber auch selbst. 
~~~



Wie in folgendem Schema ersichtlich funktioniert der Algorithmus mit einer einfachen
Schleife. Zum Zeitpunkt t = 0 wird die Population P(t) mit einer Menge d an zufälligen
Lösungen initialisiert. Anschließend wird die Fitnessfunktion für jede Lösung ausgewertet.
Mit dieser Startpopulation kann jetzt in die Schleife gestartet werden.

Bei jedem Schleifendurchlauf wird mit Hilfe der Terminierungsfunktion T festgestellt ob eine
ausreichende Lösung gefunden wurde und der Algorithmus dann passend beendet. Meistens
wird hier zusätzlich noch eine Zählvariable eingebaut die eine gewisse Maximalzahl an
Generationen vorgibt. Dadurch werden Deadlocks vermieden, und die Ausführungszeit ist
vorher abschätzbar.

Innerhalb der Schleife wird jeder Antikörper aus P(t) dup mal geklont (kopiert) und bildet die
Menge Pclo. Diese Menge der Klone Pclo wird dann mutiert. Es sind verschiedene Mutationen
möglich (siehe 3.3). Nachdem die Fitness der mutierten Antikörper Phyp ermittelt wurde, wird
aus den mutierten Antikörpern Phyp und der alten Population P(t) die neue Population P(t+1)
ausgewählt, indem die d fittesten Antikörper verwendet werden und die restlichen verworfen
werden. Alle hier angesprochene Mengen sind Multisets, d.h. sie können das gleiche Element
mehrmals enthalten. Dadurch wird sichergestellt, das besonders fitte Antikörper auch
mehrmals geklont und mutiert werden können.
 t=0
 initiate P(0) = zufällige mögliche Lösungen
 evaluate P(0)
 while ( T (P(t) ) = 0 )
   Pclo = cloning (P(t),dup)
   Phyp = hypermutation (Pclo)
   evaluate (Phyp)
   P(t+1) = select ((Phyp∪ P(t)), d)      (multiset!)
   t = t+1
 wend
 Abbildung 1 Der Algorithmus in Psydocode
3.3 Mutation
Die Art der Mutation repräsentiert unterschiedliche Suchfunktionen im Lösungsraum. Als
Mutation ist jede Map-Funktion f: {0,1} → {0,1} möglich und denkbar. Hier wird allerdings
                                               -5-
nur ein einfaches Bit flip an beliebiger, zufälliger Stelle genutzt, da wir keine zu großen
Veränderungen bewirken wollen.
3.4 Wichtige Größe: Population
Die Populationsgröße d stellt eine kritische Variable dar. Ist sie zu klein gewählt, kann es zu
einer vorzeitige Konvergenz auf eine falsche Lösung kommen, und die tatsächliche Lösung
wird dann nicht mehr gefunden.
Ist d jedoch zu groß gewählt, sind viele überflüssige Berechnungen zu machen und die
Laufzeit des Algorithmus wird unrentable. Eine Fitness-Verbesserung dauert dann zu lange.
Um diesen beiden Effekten entgegen zu wirken, wird hier nur ein einfacher Algorithmus
verwendet: ohne dynamische Populationsgröße können durch Beeinflussung von d und dub
auch nichtdominante Lösungen gefunden werden und die Dynamik der Berechnungen lassen
sich besser vorhersagen.
3.5 Nichtdominante Lösungen
Nichtdominante Lösungen werden durch einen besonderen Effekt auch ohne dynamische
Populationsgröße gefunden. Dabei wird in der „Expansions Phase“ die Population vergrößert.
Dies geschieht durch das Cloning und Hypermutation. Die Fitness Landscape wird dabei nach
möglichen Lösungen untersucht. Anschließend wird in der „Reduktions Phase“ wird die
Population dann wieder kleiner, da eine Selektion der besten Lösungen stattfindet.
\newpage
\section{Anwendung in der Informatik}
\subsection{Virenerkennung?}
\subsection{Mustererkennung?}
\subsection{Intrusion Detection Systems}
\subsection{Danger Theory?}

Genetische Algorithmen! / Klonale Selektion

\begin{thebibliography}{99}
\bibitem{Abk06} {\sc Autorname V.:}  \textit{Titel}, 
Edition, Verlag, 2006.
\bibitem{Abk06} {\sc Autorname V. 2006:}  Seite X ,,"Uberschrift / Kapitelname''
\end{thebibliography}
\end{document}
